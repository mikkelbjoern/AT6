\documentclass[12pt,a4paper]{article}
\usepackage[utf8]{inputenc}
\usepackage[danish]{babel}
\usepackage{amsmath}
\usepackage{amsfonts}
\usepackage{amssymb}
\usepackage{graphicx}
\usepackage[left=2cm,right=2cm,top=2cm,bottom=2cm]{geometry}


\usepackage{titlepic}
\usepackage{enumerate}
\usepackage{enumitem}
\usepackage{float}
\usepackage{pdfpages}
\usepackage[colorlinks = true,
            linkcolor = blue,
            urlcolor  = blue,
            citecolor = blue,
            anchorcolor = blue]{hyperref}
\usepackage[explicit]{titlesec}
\usepackage{pstricks}
\usepackage[amsmath,thmmarks]{ntheorem} %pakke til at lave sætningsenvorinmets (kan ikke loades sammen med amsthm)
\usepackage{color}
\usepackage{tikz}

%opretter environmets til sætningsstrukturen 
\theorembodyfont{\normalfont}

	
	%sætnings environment	
	\newtheorem{thm}{Sætning}

	\theoremstyle{break}	
	%opgave environment	
	\newtheorem{opg}{Opgave}	

	%Korrolar environment
	\newtheorem{korollar}[thm]{Korollar}	
	
	%Lemma environment	
	\newtheorem{lemma}[thm]{Lemma}
	
	\theoremsymbol{\ensuremath{\circ}}	
	
	%definition environment	
	\newtheorem{definition}[thm]{Definition}
	
	%eksempel environment	
	\newtheorem{eksempel}[thm]{Eksempel}
	
	
	
	%Bevis environment
	\theoremstyle{nonumberplain}
	\theoremheaderfont{%
	\normalfont\itshape}
	\theorembodyfont{\normalfont}
	\theoremsymbol{\ensuremath{\square}}
	\theoremseparator{.}
	
	\newtheorem{proof}{Bevis}
	\newtheorem{los}{Løsning}
	






\setlength\parindent{0pt}

%\titleformat{\section}{\Large\bfseries}{}{0pt}{#1}
%\titleformat{\subsection}{\large\bfseries}{}{0pt}{#1}


%nye komandoer
\newcommand{\mR}{\mathbb{R}}
\newcommand{\mZ}{\mathbb{Z}}
\newcommand{\mN}{\mathbb{N}}
\newcommand{\mQ}{\mathbb{Q}}
\newcommand{\mC}{\mathbb{C}}
\newcommand{\hs}{\hspace{2mm}}
\newcommand{\Hs}{\hspace{4mm}}
\newcommand{\pipe}{\hs | \hs}
\newcommand{\lp}{\left(}
\newcommand{\rp}{\right)}
\newcommand{\vect}[1]{\underline{#1}}
\newcommand{\matr}[1]{\underline{\underline{#1}}}
\newcommand{\cnum}[1]{\raisebox{.5pt}{\textcircled{\raisebox{-.9pt} {#1}}}}




\author{Mikkel B. Goldschmidt\\ 3r, Nørre Gymnasium \\ AT: Matematik og fysik}
\title{AT6 - Menneskets forhold til naturen \\ Deterministisk kaos}
\date{\today}



\begin{document}
\maketitle

\section{Indledning}
I 1814 skrev Laplace om en dæmon, der kendte til altings placering, fart, temperatur og alle andre fysiske og kemiske egenskaber ved verden.
Laplace argumentere for, at denne dæmon måtte kende alt fremtid og alt fortid, hvis den ellers var i stand til at kigge på alt dette data. 
Han skrev om sin idé om at en person med alt denne viden, måtte kunne beskrive verden i en matematisk ligning. 
Ultimativt skulle det være drømmen for naturvidenskaben: At vide nok til at kunne beskrive verden perfekt - og dermed vide alt.

For 50 år siden ødelagde matematikeren og metrologen Edward Lorenz dog denne drøm. 
Han havde opsat et differentialligningssystem, der kunne beskrive vejret i fremtiden ud fra en række parametre. 
Da han ikke kunne løse modellen analytisk, brugte han en numerisk løsning med en computer. 
Han prøvede dog på et tidspunkt at regne ud, hvordan vejret ville se ud om et år. 
Først regnede han et år frem i modellen og fandt et resultat. 
Han fandt dog ved et tilfælde ud af, at dette gav et fuldstændigt andet resultat end ved først at regne et halvt år frem og, derefter  regne endnu et halvt år frem, ved at bruge de tal som modellen gav. 
Årsagen til dette skulle findes i, at hans model afrundede til betydende cifre. 
Det, at nogle cifre var blevet tabt, gav altså et helt andet resultat. 
Denne relativ lille ændring i input gav altså ikke en relativ lille ændring i output. 
Han navngav dette fænomen \textit{deterministisk kaos}.
Det skulle vise sig, at deterministisk kaos var et argument for, at Laplace dæmon aldrig ville kunne eksistere i virkeligheden.

\section{Problemformulering}
I hvor høj en grad begrænser en kaotisk verden vores beskrivelse af naturen?

\begin{itemize}
\item Hvad forstås ved deterministisk kaos, og hvornår opstår det?
\item Hvordan kommer deterministisk kaos til udtryk i trelegemeproblemet?
\item Gør deterministisk kaos vores normale naturvidenskabelige praksis ubrugelig?
\item Hvordan passer det deterministiske kaos, med den menneskelige intuition om hvordan verden opfører sig?
\end{itemize}

\section{Behandling af underspørgsmål}

\subsection{Kaos}
Lorenz definerer kaos på følgende måde:
\begin{center}
		\textit{When the present determines the future, \\
		but the approximate present does not approximately determine the future.}
\end{center}
Han finder da adskillige differentialligningssystemer hvor kaos - flere end bare hans egen vejrmodeller.
Kaos er derefter blevet opdaget i mange systemer, der beskriver naturen. 

\subsection{Trelegemeproblemet}
I sin bog, Principia, skrev Newton blandt andet om sine gravitationslove. 
Han overvejede ud fra disse love hvordan forskellige objekter ville opføre sig i forhold til hinanden. 
Blandt andet finder han pæne stedfunktioner for to legemer der påvirker hinanden med gravitationskræfter. 
Det lykkes ham til gengæld ikke at gør det samme med tre legemer, og han opstiller det derfor som et problem som andre kan arbejde videre med. 

Det har senere vist sig at grunden til at det var så svært for Newton, var at trelegemeproblemet er et system hvor der opstår kaos. 
Denne opdagelse er ret central, for det udelukker faktisk at vi nogensinde vil kunne forudsige naturen længere ud i fremtiden, da der i langt de fleste systemer er minimum tre legemer (det kan enten være planeter, elektroner eller luftmolekyler - problemet opstår alle steder).

Ved at indse at der opstår kaos i alle systemer med minimum tre legemer, bliver vi faktisk også nødt til at indse, at naturen er umulig at forudsige, så længe vi ikke kan måle og regne uendelig præcist.

\subsection{Videnskabelig praksis i en kaotisk verden}
Når vi laver naturvidenskab, er det ofte vores mål, at få opstillet regler der er så generelle, at de kan bruges til at forudsige verden med. 
Med tilstrækkelige oplysninger, mener vi at kunne give en nogenlunde beskrivelse af hvad der kommer til at ske med verden lige om lidt.
Eksempelvis er de fleste stærkt overbeviste om at når du giver slip på en blyant, så falder den til gulvet. 
Lidt forudsigelighed har vi altså i verden. 

En kritiker kunne så spørge hvorfor kaos overhovedet er relevant. 
Selvom vi kan lave forudsigelser om verden på kort sigt, så opstår der problemer på længere sigt. 
Vi oplever det dagligt efter nyhederne om aftenen, når metrologen går på. 
Metrologerne lever at at forudsige verden et stykke frem i tiden, men alle ved også godt at de tager fejl ofte og at man aldrig kan regne med dem mere en et par dage frem. 
Det er ikke fordi vores metrologerne er inkompetente, de arbejder bare med et kaotisk system - naturen.
Det er klart nemt for dem at sige hvordan vejret ser ud om en time, og de vil kunne sige dem med forholdsvist stor nøjagtighed.
Men spørger man dem om hvordan vejret ser ud om præcis 4 måneder, ved de godt at de ikke har en chance. 
De kan nemlig kun regne på approksimationer af virkeligheden, og i kaotiske systemer, giver disse ikke en approksimation af virkeligheden.

Vi må derfor indse at naturvidenskaben aldrig kommer til at forudsige verden længere frem i tiden. 

\subsection{Den menneskelige intuition om en kaotisk verden}
Laplace dæmon, som beskrevet i indledningen, beskriver nok den intuition som mange mennesker har når de starter på at lave naturvidenskab.
Lige pludselig er vi i stand til at forudsige hvad der sker, og man fristes derfor til at tænke som Laplace, at vi vil kunne forudsige alt hvad der sker i verden.

Med deterministisk kaos må vi indse, at dette ikke er muligt. 
Vi ville kun kunne det hvis vi kunne måle uendelig præcist og regne lige så præcist, hvilket oplagt ikke er muligt.

Vi kan dog godt lave approksimationer af virkeligheden over kortere tid der er forholdsvist præcise. 

Dermed er naturvidenskaben stadig brugbar, men vi er ikke i stand til at forudsige hvordan naturen vil opføre sig præcist.


\section{Fagenes metoder}
\subsection{Fysik}
Fysik har i behandlingen af mine underspørgsmål primært været brugt til at beskrive virkeligheden. 
Jeg har udnyttet at fysik som naturvidenskabeligt fag kan beskrive verden omkring os.
Det er fysikken der er blevet brugt til at opstille modeller omkring virkeligheden, specifikt er det her blevet brugt til at opstille en model over trelegemeproblemet.

\subsection{Matematik}
I behandlingen af underspørgsmålene har matematik mest været brugt anvendt til at kigge på de sammenhænge som fysikken har opstillet. 
Matematikken kan ikke sige noget direkte om verden, men kan, givet et system der følger faste regler, sige ting om systemet. 
Det er på denne måde matematikken her har været udnyttet. 
Ved at tage et system opstillet med fysiske love, har med matematikken kunnet vise at kaos opstår. 

 
\end{document}