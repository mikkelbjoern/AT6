\documentclass[10pt,a4paper]{article}
\usepackage[utf8]{inputenc}
\usepackage[danish]{babel}
\usepackage{amsmath}
\usepackage{amsfonts}
\usepackage{amssymb}
\usepackage{graphicx}
\usepackage[left=2cm,right=2cm,top=2cm,bottom=2cm]{geometry}


\usepackage{titlepic}
\usepackage{enumerate}
\usepackage{enumitem}
\usepackage{float}
\usepackage{pdfpages}
\usepackage[colorlinks = true,
            linkcolor = blue,
            urlcolor  = blue,
            citecolor = blue,
            anchorcolor = blue]{hyperref}
\usepackage[explicit]{titlesec}
\usepackage{pstricks}
\usepackage[amsmath,thmmarks]{ntheorem} %pakke til at lave sætningsenvorinmets (kan ikke loades sammen med amsthm)
\usepackage{color}
\usepackage{tikz}

%opretter environmets til sætningsstrukturen 
\theorembodyfont{\normalfont}

	
	%sætnings environment	
	\newtheorem{thm}{Sætning}

	\theoremstyle{break}	
	%opgave environment	
	\newtheorem{opg}{Opgave}	

	%Korrolar environment
	\newtheorem{korollar}[thm]{Korollar}	
	
	%Lemma environment	
	\newtheorem{lemma}[thm]{Lemma}
	
	\theoremsymbol{\ensuremath{\circ}}	
	
	%definition environment	
	\newtheorem{definition}[thm]{Definition}
	
	%eksempel environment	
	\newtheorem{eksempel}[thm]{Eksempel}
	
	
	
	%Bevis environment
	\theoremstyle{nonumberplain}
	\theoremheaderfont{%
	\normalfont\itshape}
	\theorembodyfont{\normalfont}
	\theoremsymbol{\ensuremath{\square}}
	\theoremseparator{.}
	
	\newtheorem{proof}{Bevis}
	\newtheorem{los}{Løsning}
	






\setlength\parindent{0pt}

%\titleformat{\section}{\Large\bfseries}{}{0pt}{#1}
%\titleformat{\subsection}{\large\bfseries}{}{0pt}{#1}


%nye komandoer
\newcommand{\mR}{\mathbb{R}}
\newcommand{\mZ}{\mathbb{Z}}
\newcommand{\mN}{\mathbb{N}}
\newcommand{\mQ}{\mathbb{Q}}
\newcommand{\mC}{\mathbb{C}}
\newcommand{\hs}{\hspace{2mm}}
\newcommand{\Hs}{\hspace{4mm}}
\newcommand{\pipe}{\hs | \hs}
\newcommand{\lp}{\left(}
\newcommand{\rp}{\right)}
\newcommand{\vect}[1]{\underline{#1}}
\newcommand{\matr}[1]{\underline{\underline{#1}}}
\newcommand{\cnum}[1]{\raisebox{.5pt}{\textcircled{\raisebox{-.9pt} {#1}}}}




\author{Mikkel B. Goldschmidt\\ 3r, Nørre Gymnasium \\ AT: Matematik og fysik}
\title{AT6 - Menneskets forhold til naturen \\ Deterministisk kaos}
\date{\today}



\begin{document}
\maketitle

\section{Indledning} For nylig kiggede jeg ind af døren til en 1.g-klasse, der havde fysik. 
De var ved at bestemme jordens tyngdeacceleration ved at måle med et stopur, hvor hurtigt et objekt faldt til jorden. 
Da jeg selv lavede samme forsøg, var jeg godt klar over, at vi nok ikke kunne trykke på stopuret på præcis det tidspunkt, hvor objektet ramte jorden - måske ville vi ramme et halvt sekund for tidligt, eller et halvt sekund for sent. 
Det tænkte vi dog ikke det store over, for vores resultat ville ikke variere så meget, så det betød nok ikke det store. 
Hvis de regnede videre med den værdi, de selv havde bestemt i stedet for værdien fra deres databog, ville de nok ikke få markant anderledes resultater. 
Tanken bag her har virket ret intuitiv for videnskabsmænd gennem tiden: 
\textit{En relativ lille ændring i input giver en relativ lille ændring i output}. 
Dette må nødvendigvis være sandt, hvis vi skal kunne bruge målinger, der indeholder små fejl.

Meteorologen og matematikeren Edward Lorenz beskrev i 1963, at denne antagelse ikke altid passer. 
Han havde opsat et differentialligningssystem, der kunne beskrive vejret i fremtiden ud fra en række parametre. 
Da han ikke kunne løse modellen analystisk, brugte han en numerisk løsning med en computer. 
Han prøvede dog på et tidspunkt at regne ud, hvordan vejret ville se ud om et år. 
Først regnede han et år frem i modellen og fandt et resultat. 
Han fandt dog ved et tilfælde ud af, at dette gav et fuldstændigt andet resultat end først at regne et halvt år frem og derefter at bruge de tal som modellen gav til at regne et halvt år mere frem. 
Årsagen til dette skulle findes i, at hans model afrundede til betydende cifre. 
Det, at nogle cifre var blevet tabt, gav altså et helt andet resultat. 
Denne relativ lille ændring i input gav altså ikke en relativ lille ændring i output. 
Han navngav dette fænomen deterministisk kaos.



\end{document}