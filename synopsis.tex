\documentclass[12pt,a4paper]{article}
\usepackage[utf8]{inputenc}
\usepackage[danish]{babel}
\usepackage{amsmath}
\usepackage{amsfonts}
\usepackage{amssymb}
\usepackage{graphicx}
\usepackage[left=2cm,right=2cm,top=2cm,bottom=2cm]{geometry}


\usepackage{titlepic}
\usepackage{enumerate}
\usepackage{enumitem}
\usepackage{float}
\usepackage{pdfpages}
\usepackage[colorlinks = true,
            linkcolor = blue,
            urlcolor  = blue,
            citecolor = blue,
            anchorcolor = blue]{hyperref}
\usepackage[explicit]{titlesec}
\usepackage{pstricks}
\usepackage[amsmath,thmmarks]{ntheorem} %pakke til at lave sætningsenvorinmets (kan ikke loades sammen med amsthm)
\usepackage{color}
\usepackage{tikz}

%opretter environmets til sætningsstrukturen 
\theorembodyfont{\normalfont}

	
	%sætnings environment	
	\newtheorem{thm}{Sætning}

	\theoremstyle{break}	
	%opgave environment	
	\newtheorem{opg}{Opgave}	

	%Korrolar environment
	\newtheorem{korollar}[thm]{Korollar}	
	
	%Lemma environment	
	\newtheorem{lemma}[thm]{Lemma}
	
	\theoremsymbol{\ensuremath{\circ}}	
	
	%definition environment	
	\newtheorem{definition}[thm]{Definition}
	
	%eksempel environment	
	\newtheorem{eksempel}[thm]{Eksempel}
	
	
	
	%Bevis environment
	\theoremstyle{nonumberplain}
	\theoremheaderfont{%
	\normalfont\itshape}
	\theorembodyfont{\normalfont}
	\theoremsymbol{\ensuremath{\square}}
	\theoremseparator{.}
	
	\newtheorem{proof}{Bevis}
	\newtheorem{los}{Løsning}
	






\setlength\parindent{0pt}

%\titleformat{\section}{\Large\bfseries}{}{0pt}{#1}
%\titleformat{\subsection}{\large\bfseries}{}{0pt}{#1}


%nye komandoer
\newcommand{\mR}{\mathbb{R}}
\newcommand{\mZ}{\mathbb{Z}}
\newcommand{\mN}{\mathbb{N}}
\newcommand{\mQ}{\mathbb{Q}}
\newcommand{\mC}{\mathbb{C}}
\newcommand{\hs}{\hspace{2mm}}
\newcommand{\Hs}{\hspace{4mm}}
\newcommand{\pipe}{\hs | \hs}
\newcommand{\lp}{\left(}
\newcommand{\rp}{\right)}
\newcommand{\vect}[1]{\underline{#1}}
\newcommand{\matr}[1]{\underline{\underline{#1}}}
\newcommand{\cnum}[1]{\raisebox{.5pt}{\textcircled{\raisebox{-.9pt} {#1}}}}




\author{Mikkel B. Goldschmidt\\ 3r, Nørre Gymnasium \\ AT: Matematik og fysik}
\title{AT6 - Menneskets forhold til naturen \\ Deterministisk kaos}
\date{\today}



\begin{document}
\maketitle

\section{Indledning} For nylig kiggede jeg ind af døren til en 1.g-klasse, der havde fysik. 
De var ved at bestemme jordens tyngdeacceleration ved at måle med et stopur, hvor hurtigt et objekt faldt til jorden. 
Da jeg selv lavede samme forsøg, var jeg godt klar over, at vi nok ikke kunne trykke på stopuret på præcis det tidspunkt, hvor objektet ramte jorden - måske ville vi ramme et halvt sekund for tidligt, eller et halvt sekund for sent. 
Det tænkte vi dog ikke det store over, for vores resultat ville ikke variere så meget, så det betød nok ikke det store. 
Hvis de regnede videre med den værdi, de selv havde bestemt i stedet for værdien fra deres databog, ville de nok ikke få markant anderledes resultater. 
Tanken bag her har virket ret intuitiv for videnskabsmænd gennem tiden: 
\textit{En relativ lille ændring i input giver en relativ lille ændring i output}. 
Dette må nødvendigvis være sandt, hvis vi skal kunne bruge målinger, der indeholder små fejl.

Meteorologen og matematikeren Edward Lorenz beskrev i 1963, at denne antagelse ikke altid passer. 
Han havde opsat et differentialligningssystem, der kunne beskrive vejret i fremtiden ud fra en række parametre. 
Da han ikke kunne løse modellen analytisk, brugte han en numerisk løsning med en computer. 
Han prøvede dog på et tidspunkt at regne ud, hvordan vejret ville se ud om et år. 
Først regnede han et år frem i modellen og fandt et resultat. 
Han fandt dog ved et tilfælde ud af, at dette gav et fuldstændigt andet resultat end ved først at regne et halvt år frem og, derefter  regne endnu et halvt år frem, ved at bruge de tal som modellen gav. 
Årsagen til dette skulle findes i, at hans model afrundede til betydende cifre. 
Det, at nogle cifre var blevet tabt, gav altså et helt andet resultat. 
Denne relativ lille ændring i input gav altså ikke en relativ lille ændring i output. 
Han navngav dette fænomen \textit{deterministisk kaos}.

\section{Problemformulering}
I hvor høj en grad begrænser en kaotisk verden vores beskrivelse af naturen?

\begin{itemize}
\item Hvad forstås ved deterministisk kaos, og hvornår opstår det?
\item Hvordan kommer deterministisk kaos til udtryk i trelegemeproblemet?
\item Gør deterministisk kaos vores normale naturvidenskabelige praksis ubrugelig?
\item Hvordan passer det deterministiske kaos, med den menneskelige intuition om hvordan verden opfører sig?
\end{itemize}

\section{Behandling af underspørgsmål}

\subsection{Kaos}
Lorenz definerer kaos på følgende måde:
\begin{center}
		\textit{When the present determines the future, \\
		but the approximate present does not approximately determine the future.}
\end{center}
Han finder da adskillige differentialligningssystemer hvor kaos - flere end bare hans egen vejrmodeller.
Kaos er derefter blevet opdaget i mange systemer, der beskriver naturen. 

\subsection{Trelegemeproblemet}
I sin bog, Principia, skrev Newton blandt andet om sine gravitationslove. 
Han overvejede ud fra disse love hvordan forskellige objekter ville opføre sig i forhold til hinanden. 
Blandt andet finder han pæne stedfunktioner for to legemer der påvirker hinanden med gravitationskræfter. 
Det lykkes ham til gengæld ikke at gør det samme med tre legemer, og han opstiller det derfor som et problem som andre kan arbejde videre med. 

Det har senere vist sig at grunden til at det var så svært for Newton, var at trelegemeproblemet er et system hvor der opstår kaos. 
Denne opdagelse er ret central, for det udelukker faktisk at vi nogensinde vil kunne forudsige naturen længere ud i fremtiden, da der i langt de fleste systemer er minimum tre legemer (det kan enten være planeter, elektroner eller luftmolekyler - problemet opstår alle steder).

Ved at indse at der opstår kaos i alle systemer med minimum tre legemer, bliver vi faktisk også nødt til at indse, at naturen er umulig at forudsige, så længe vi ikke kan måle og regne uendelig præcist.


\end{document}