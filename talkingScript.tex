\documentclass[12pt,a4paper]{article}
\usepackage[utf8]{inputenc}
\usepackage[danish]{babel}
\usepackage{amsmath}
\usepackage{amsfonts}
\usepackage{amssymb}
\usepackage{graphicx}
\usepackage[left=2cm,right=2cm,top=2cm,bottom=2cm]{geometry}


\usepackage{titlepic}
\usepackage{enumerate}
\usepackage{enumitem}
\usepackage{float}
\usepackage{pdfpages}
\usepackage[colorlinks = true,
            linkcolor = blue,
            urlcolor  = blue,
            citecolor = blue,
            anchorcolor = blue]{hyperref}
\usepackage[explicit]{titlesec}
\usepackage{pstricks}
\usepackage[amsmath,thmmarks]{ntheorem} %pakke til at lave sætningsenvorinmets (kan ikke loades sammen med amsthm)
\usepackage{color}
\usepackage{tikz}

%opretter environmets til sætningsstrukturen 
\theorembodyfont{\normalfont}

	
	%sætnings environment	
	\newtheorem{thm}{Sætning}

	\theoremstyle{break}	
	%opgave environment	
	\newtheorem{opg}{Opgave}	

	%Korrolar environment
	\newtheorem{korollar}[thm]{Korollar}	
	
	%Lemma environment	
	\newtheorem{lemma}[thm]{Lemma}
	
	\theoremsymbol{\ensuremath{\circ}}	
	
	%definition environment	
	\newtheorem{definition}[thm]{Definition}
	
	%eksempel environment	
	\newtheorem{eksempel}[thm]{Eksempel}
	
	
	
	%Bevis environment
	\theoremstyle{nonumberplain}
	\theoremheaderfont{%
	\normalfont\itshape}
	\theorembodyfont{\normalfont}
	\theoremsymbol{\ensuremath{\square}}
	\theoremseparator{.}
	
	\newtheorem{proof}{Bevis}
	\newtheorem{los}{Løsning}
	






\setlength\parindent{0pt}

%\titleformat{\section}{\Large\bfseries}{}{0pt}{#1}
%\titleformat{\subsection}{\large\bfseries}{}{0pt}{#1}


%nye komandoer
\newcommand{\mR}{\mathbb{R}}
\newcommand{\mZ}{\mathbb{Z}}
\newcommand{\mN}{\mathbb{N}}
\newcommand{\mQ}{\mathbb{Q}}
\newcommand{\mC}{\mathbb{C}}
\newcommand{\hs}{\hspace{2mm}}
\newcommand{\Hs}{\hspace{4mm}}
\newcommand{\pipe}{\hs | \hs}
\newcommand{\lp}{\left(}
\newcommand{\rp}{\right)}
\newcommand{\vect}[1]{\underline{#1}}
\newcommand{\matr}[1]{\underline{\underline{#1}}}
\newcommand{\cnum}[1]{\raisebox{.5pt}{\textcircled{\raisebox{-.9pt} {#1}}}}




\author{Mikkel B. Goldschmidt\\ 3r, Nørre Gymnasium \\ AT: Matematik og fysik}
\title{AT6 - Menneskets forhold til naturen \\ Deterministisk kaos}
\date{\today}


\usepackage{csquotes}

\begin{document}
\maketitle
Her til morgen da jeg var på vej ud af døren, glemte jeg min mobil inde på bordet. 
Jeg løb derfor ind og hentede den - det tog ikke mere end måske 30 sekunder. 
Efter at have hentet mobilen tog jeg cyklen og cyklede afsted imod skolen som jeg ellers ville have gjort. 
På vej opad bakken ude foran skolen så jeg så en bil der kørte med minimum 90 i timen -  den havde helt sikkert ikke kunnet nå at stoppe. 
Havde jeg ikke glemt min telefon, så kunne jeg være endt foran den bil og mit liv ville have ændret sig for evigt.
Så en forholdvist ændring i mit liv ender med over tid at gøre en kæmpe forskel. 
Dette er ikke en antagelse vi normalt arbejder med i naturvidenskab. 
Udfører vi et forsøg med 0.5 K højere temperatur forventer vi ikke at forsøgets udfald ændrer sig markant. 
Men verden ser ikke ud til at passe så godt på denne optagelse. 
Det blev påvist af en meteorolog ved navn Lorenz. 
Han viste i nogle af sine vejrmodeller at ændringer så små som afrunding til betydende cifre kunne have afgørende indflydelse på modellens forudsigelser. 
Dette fænomen hvor en forholdsvist lille ændring har stor betydning kaldte han for "kaos". 

Jeg vil i denne fremlæggelse snakke om hvilken betydning dette kaos har for naturvidenskab - for kan vi overhovedet tillade os som videnskabsmænd at lave forudsigelser om fremtiden med modeller der indeholder kaos. 
Jeg vil derudover fremvise min påvisning af kaos i Newtons gravitationslove.

Kaos er som sagt en egenskab et system kan have hvor små ændringer over tid gør en stor forskel. 
Klassiske eksempler på dette er dobbelt penduler, billiardborde, vejrmodeller,.. mm.
Jeg har valgt at kigge på Newtons gravitationslove, da jeg synes at disse er meget nemme at pege på i virkeligheden omkring os. 
Specifikt har jeg opsat en differentiallignigsmodel der beskriver trelegemeproblemet med Newtons gravitationslove. 
Jeg har vist skrevet i min synopse at dette system ikke har en analystiks løsning - dette er forkert, den er bare en afsindig grim uendelig konvergerende sum, hvorfor det ikke i praksis er nogen særlig farbar vej at regne ud fra. 
Jeg har derfor valgt at udnytte numerisk metode til at løse differentialligningssystemet. 
I min synopse er min model beskrevet, så jeg vil ikke gå vildt meget i detaljer med de tekniske detaljer med mindre I spørger ind til det. 
Men opsummeret har jeg opstillet en differentialligningsmodel der beskriver 3 objekter der påvirker hinanden med gravitation. 
De har alle sammen en tilhørende masse og to todimensionelle vektorer der beskriver hhv. hastighed og sted. 
Jeg har da kørt to systemer uafhængigt af hinanden oveni hinanden med næsten samme startbetingelser - det ene system havde bare fået rykket et legeme mindre end en 10000-del pixel i forhold til det tilsvarende legeme i det andet system.

\begin{center}
    --- Start simulation ---
\end{center}

Vi kan her kun se det ene system. 
Det skyldes at de to systemer har næsten de samme placeringer på skærmen, og derfor kan man kun se det ene.
Ind til videre ser det jo meget godt ud - de opfører sig ens og det ser derfor ikke ud som om, at den lille ændring i startbetingelser har indflydelse. 
Men prøv nu at lægge mærke til de små legemer.
Bag det ene af dem kan man lige se lidt hvidt der kigger frem. 
Meget hurtigt ser vi kaos. 
Den lille bitte ændring i startpositionen, ender med at have meget stor virkning på systemet efter forholdsvist kort tid. 
Da dette system er bygget på Newtons gravitationslove, må de samme egenskaber gælde for vores solsystem og universet generelt. 
Det må desuden gælde for alle andre objekter med masse, hvorfor man kunne tænke at vi intet kan forudsige hvis vi bare har den mindste smule usikkerhed i vores observationer. 
Og dette resultat stiller så det meget fundamentale spørgsmål, om vi overhovedet kan stole på forudsigelser lavet med naturvidenskab.

Dog ved vi at vi kan lave mange meget præcise forudsigelser. 
Vi kan forudsige ned til sekunders præcision hvornår der kommer solformørkelse, og præcis hvor det sker.
Derfor kan vi jo også sige at vores forudsigelser ikke er helt uduelige. 
Den store forskel er hvor lang tid ude i fremtiden vi vil forudsige ting, og hvor mange variable der indgår.
Når først vi snakker om meget lang tid ude i fremtiden, så er alt hvad vi forudsiger meget usandsynligt.
Vi bliver derfor nødt til at acceptere at der er begrænsninger for hvad vi kan forudsige med videnskaben.

Dette spiller ikke så godt sammen med den intuition vi mennesker har om naturen.
Jeg ved i hvert fald at jeg selv havde en tro på at naturen var deterministisk, da jeg selv startede i gymnasiet - helt frem til at jeg for første gang blev introduceret til tilfældighederne i kvantemekanikken. 

Det er så her min synopse sluttede. 
Jeg har siden kigget videre og til min store lettelse fundet nyt forskning på området.
Selvom kaos ser helt forfærdeligt usystematisk ud, er man begyndt at kunne finde systemer i hvordan kaos opfører sig. 
Man har indført systemkonstanter der beskriver hvor sensitive kaotiske systemer er overfor ændringer. 
Jeg forstod ikke den matematik der blev beskrevet i artiklen, så jeg gik i gang selv. 
Jeg lavede et forsøg på mit system hvor jeg målte den tid det tog for de to systemer at blive tydeligt forskellige på skærmen i forhold til hvor meget den ene planet var forskudt. 
Jeg har derefter forsøgt at finde en model der kunne beskrive en sammenhæng. 
Jeg fandt en model på formen:
$$ t(d) = k \cdot \frac{1}{d^n} - k $$

Det er helt sikkert ikke nogen perfekt model, og den ligner heller ikke det jeg så i de artikler jeg læste. 
De gjorde dog brug af funktioner jeg ikke forstod, og jeg valgte derfor at gøre forsøget selv.
Det resultat er man kan finde lidt orden i dette kaotiske system er lidt tilfredsstillende da det passer ind i min menneskelige intuition om at der er orden i naturen. 

Overordnet set har jeg i dette projekt fundet at der er en begrænsning for hvor præcist vi mennesker kan forudsige hvordan naturen opfører sig. 
Vi ser dette direkte ud af det kaos jeg har påvist i ved hjælp af Newtons love i trelegemeproblemet. 
Dette passer ikke så godt ind i den menneskelige intuition om at verden kan forudsiges hvis man kender til nok parametre på forhånd. 
Heldigvis har jeg også fundet at vi kan forudsige ting om kaos, hvorfor vores naturvidenskab ikke er ubrugelig, men kun begrænset. 

Jeg har i dette AT-forløb brugt matematik som et redskab i fysikken. 
Fysikken har jeg brugt til at opstille en model ud fra fysiske love som andre har opstillet.
Denne model har jeg da kunnet behandle med matematikken hvilket har givet mig oplysninger om det fysiske fænomen jeg har forsøgt at beskrive.

Dette forløb ligner lidt det forrige AT-førløb jeg lavede omhandlende sygdomsspredning hvor jeg også anvendte numerisk metode til at lave nogle forudsigelser i en differentialligningsmodel. 
Desuden har den lighed i forhold til behandlet emne med det første forløb jeg havde omkring verdensbilleder hvor vi bla. snakkede om forudsigelser at planeters placeringer på himlen.






\end{document}